%
% Template Laporan Skripsi/Thesis
%
% @author  Andreas Febrian, Lia Sadita
% @version 1.03
%
% Dokumen ini dibuat berdasarkan standar IEEE dalam membuat class untuk
% LaTeX dan konfigurasi LaTeX yang digunakan Fahrurrozi Rahman ketika
% membuat laporan skripsi. Konfigurasi yang lama telah disesuaikan dengan
% aturan penulisan thesis yang dikeluarkan UI pada tahun 2008.
%

%
% Tipe dokumen adalah report dengan satu kolom.
%
\documentclass[12pt, a4paper, onecolumn, twoside, final]{report}
\raggedbottom

% Load konfigurasi LaTeX untuk tipe laporan thesis
\usepackage{uithesis}


% Load konfigurasi khusus untuk laporan yang sedang dibuat
%-----------------------------------------------------------------------------%
% Informasi Mengenai Dokumen
%-----------------------------------------------------------------------------%
%
% Judul laporan.
\var{\judul}{Judul Karya Ilmiah Anda}
%
% Tulis kembali judul laporan, kali ini akan diubah menjadi huruf kapital
\Var{\Judul}{Judul Karya Ilmiah Anda}
%
% Tulis kembali judul laporan namun dengan bahasa Ingris
\var{\judulInggris}{Your Scientific Publication Title}

%
% Tipe laporan, dapat berisi Skripsi, Tugas Akhir, Thesis, atau Disertasi
\var{\type}{Skripsi}
%
% Tulis kembali tipe laporan, kali ini akan diubah menjadi huruf kapital
\Var{\Type}{Skripsi}
%
% Tulis nama penulis
\var{\penulis}{Nama Lengkap Anda}
%
% Tulis kembali nama penulis, kali ini akan diubah menjadi huruf kapital
\Var{\Penulis}{Nama Lengkap Anda}
%
% Tulis NPM penulis
\var{\npm}{Nomor Anda}
%
% Tuliskan Fakultas dimana penulis berada
\Var{\Fakultas}{Fakultas Anda}
\var{\fakultas}{Fakultas Anda}
%
% Tuliskan Program Studi yang diambil penulis
\Var{\Program}{Jurusan Anda}
\var{\program}{Jurusan Anda}
%
% Tuliskan tahun publikasi laporan
\Var{\bulanTahun}{Bulan Tahun}
%
% Tuliskan gelar yang akan diperoleh dengan menyerahkan laporan ini
\var{\gelar}{Gelar Jurusan Anda}
%
% Tuliskan tanggal pengesahan laporan, waktu dimana laporan diserahkan ke
% penguji/sekretariat
\var{\tanggalSiapSidang}{Tanggal Bulan Tahun}
%
% Tuliskan tanggal keputusan sidang dikeluarkan dan penulis dinyatakan
% lulus/tidak lulus
\var{\tanggalLulus}{Tanggal Bulan Tahun}
% Tuliskan tanggal pengesahan laporan final, waktu dimana laporan
% diserahkan ke perpustakaan
\var{\tanggalFinal}{Tanggal Bulan Tahun}
%
% Tuliskan pembimbing
\var{\pembimbingSatu}{Pembimbing Pertama Anda}
\var{\pembimbingDua}{Pembimbing Kedua Anda}
%
% Tuliskan penguji
\var{\pengujiSatu}{Penguji Pertama Anda}
\var{\pengujiDua}{Penguji Kedua Anda}

%-----------------------------------------------------------------------------%
% Judul Setiap Bab
%-----------------------------------------------------------------------------%
%
% Berikut ada judul-judul setiap bab.
% Silahkan diubah sesuai dengan kebutuhan.
%
\Var{\kataPengantar}{Kata Pengantar}
\Var{\babSatu}{Pendahuluan}
\Var{\babDua}{Kerangka Berpikir}
\Var{\babTiga}{Judul Bab 3}
\Var{\babEmpat}{Judul Bab 4}
\Var{\babLima}{Judul Bab 5}
\Var{\kesimpulan}{Penutup}

% Daftar pemenggalan suku kata dan istilah dalam LaTeX
\include{hype.indonesia}
% Daftar istilah yang mungkin perlu ditandai
\input{istilah}

% Awal bagian penulisan laporan
\begin{document}

%
% Sampul Laporan
\include{sampul}
\forceclearchapter

%
% Gunakan penomeran romawi
\pagenumbering{roman}
\pagestyle{first-pages}

%
% load halaman judul dalam
\addChapter{HALAMAN JUDUL}
\include{judul_dalam}
\forceclearchapter

%
% setelah bagian ini, halaman dihitung sebagai halaman ke 2
\setcounter{page}{2}

%
% load halaman pengesahan
\addChapter{LEMBAR PERSETUJUAN}
\include{pengesahan}
\forceclearchapter
%
% load halaman orisinalitas
\addChapter{LEMBAR PERNYATAAN ORISINALITAS}
\include{orisinal}
\forceclearchapter
%
%
\addChapter{LEMBAR PENGESAHAN}
%
% Halaman Pengesahan Sidang
%
% @author  Andreas Febrian, Andre Tampubolon
% @version 1.02
%

\chapter*{HALAMAN PENGESAHAN}

\vspace*{0.4cm}
\noindent

\noindent
\begin{tabular}{ll p{9cm}}
	\type~ini diajukan oleh&: & \\
	Nama&: & \penulis \\
	NPM&: & \npm \\
	Program Studi&: & \program \\
	Judul \type&: & \judul \\
\end{tabular} \\

\vspace*{1.0cm}

\noindent \bo{Telah berhasil dipertahankan di hadapan Dewan Penguji
dan diterima sebagai bagian persyaratan yang diperlukan untuk
memperoleh gelar \gelar~pada Program Studi \program, Fakultas
\fakultas, Universitas Indonesia.}\\[0.2cm]

\begin{center}
	\bo{DEWAN PENGUJI}
\end{center}

\vspace*{0.3cm}

\begin{tabular}{l l l l }
	& & & \\
	Pembimbing 1&: & \pembimbingSatu & (\hspace*{3.0cm}) \\
	& & & \\
	Pembimbing 2&: & \pembimbingDua & (\hspace*{3.0cm}) \\
	& & & \\
	Penguji 1&: & \pengujiSatu & (\hspace*{3.0cm}) \\
	& & & \\
	Penguji 2&: & \pengujiDua & (\hspace*{3.0cm}) \\
\end{tabular}\\

\vspace*{2.0cm}

\begin{tabular}{ll l}
	Ditetapkan di&: & Depok\\
	Tanggal&: & \tanggalLulus \\
\end{tabular}


\newpage

\forceclearchapter
%
%
\addChapter{LEMBAR PERSETUJUAN PUBLIKASI ILMIAH}
\include{persetujuan_publikasi}
\forceclearchapter
%
%
\addChapter{\kataPengantar}
%-----------------------------------------------------------------------------%
\chapter*{\kataPengantar}
%-----------------------------------------------------------------------------%
Pendahuluan.
Ucapan Terima Kasih:
\begin{enumerate}[topsep=0pt,itemsep=-1ex,partopsep=1ex,parsep=1ex]
	\item Pembimbing.
	\item Dosen.
	\item Instansi.
	\item Orang tua.
	\item Sahabat.
	\item Teman.
\end{enumerate}

Penulis menyadari bahwa laporan \type~ini masih jauh dari sempurna. Oleh karena itu, apabila terdapat kesalahan atau kekurangan dalam laporan ini, Penulis memohon agar kritik dan saran bisa disampaikan langsung melalui \f{e-mail} \code{emailanda@mail.id}.

\vspace*{0.1cm}
\begin{flushright}
Depok, \tanggalSiapSidang\\[0.1cm]
\vspace*{1cm}
\penulis

\end{flushright}

%
%
\addChapter{ABSTRAK}
%
% Halaman Abstrak
%
% @author  Andreas Febrian
% @version 1.00
%

\chapter*{Abstrak}
\singlespacing

\vspace*{0.2cm}

\noindent \begin{tabular}{l l p{10cm}}
	Nama&: & \penulis \\
	Program Studi&: & \program \\
	Judul&: & \judul \\
\end{tabular} \\

\vspace*{0.5cm}

\noindent Isi abstrak. \\

\vspace*{0.2cm}

\noindent \bo{Kata kunci:} \\ \f{Keyword} satu, kata kunci dua \\

\onehalfspacing
\newpage

%
%
%
% Halaman Abstract
%
% @author  Andreas Febrian
% @version 2.1.2
% @edit by Ichlasul Affan
%

\chapter*{ABSTRACT}
\singlespacing

\vspace*{0.2cm}

% Untuk conditional statement pembimbing dua
\def\blank{}

\noindent \begin{tabular}{l l p{11.0cm}}
	Name&: & \penulis \\
	Study Program&: & \studyProgram \\
	Title&: & \judulInggris \\
	Counsellor&: & \pembimbingSatu \\
	\ifx\blank\pembimbingDua
	\else
		\ &\ & \pembimbingDua \\
	\fi
	\ifx\blank\pembimbingTiga
	\else
		\ &\ & \pembimbingTiga \\
	\fi
\end{tabular} \\

\vspace*{0.5cm}

\noindent Abstract content. \\

\vspace*{0.2cm}

\noindent Key words: \\ Keyword one, keyword two \\

\setstretch{1.4}
\newpage


%
% Daftar isi, gambar, tabel, dan kode
%
\phantomsection %hack to make them clickable
\singlespacing
\tableofcontents
\onehalfspacing
\clearpage
\phantomsection %hack to make them clickable
\singlespacing
\listoffigures
\onehalfspacing
\clearpage
\phantomsection %hack to make them clickable
\singlespacing
\listoftables
\onehalfspacing
\clearpage
\phantomsection %hack to make them clickable
\addcontentsline{toc}{chapter}{\lstlistlistingname}
\singlespacing
\lstlistoflistings
\onehalfspacing

% Jika penomoran romawi selesai di ganjil
\naiveoddclearchapter
% Jika penomoran romawi selesai di genap
%\naiveevenclearchapter

%
% Gunakan penomeran Arab (1, 2, 3, ...) setelah bagian ini.
%
\pagenumbering{arabic}
\pagestyle{standard}
% \setlength{\belowcaptionskip}{+2pt}


\setoddevenheader
%-----------------------------------------------------------------------------%
\chapter{\babSatu}
%-----------------------------------------------------------------------------%
Pada bab ini, akan dijelaskan tentang latar belakang dan permasalahan yang diselesaikan pada penelitian ini.


%-----------------------------------------------------------------------------%
\section{Latar Belakang}
%-----------------------------------------------------------------------------%
Tentukan latar belakang dari penelitian Anda di sini (\f{background}).

%-----------------------------------------------------------------------------%
\section{Permasalahan}
%-----------------------------------------------------------------------------%
Sebutkan permasalahan penelitian Anda dari latar belakang tersebut.

%-----------------------------------------------------------------------------%
\subsection{Definisi Permasalahan}
%-----------------------------------------------------------------------------%
Berikut ini adalah rumusan permasalahan dari penelitian yang dilakukan:
\begin{itemize}
	\item Bagaimana cara membuat pertanyaan penelitian?
\end{itemize}


%-----------------------------------------------------------------------------%
\subsection{Batasan Permasalahan}
%-----------------------------------------------------------------------------%
Berikut ini adalah asumsi yang digunakan sebagai batasan penelitian ini:
\begin{itemize}
	\item Salah satu batasannya adalah, ini hanya \f{template}.
\end{itemize}


%-----------------------------------------------------------------------------%
\section{Tujuan Penelitian}
%-----------------------------------------------------------------------------%
Berikut ini adalah tujuan penelitian yang dilakukan:
\begin{itemize}
	\item Untuk memberikan \f{template} yang dapat mempermudah skripsi orang lain.
\end{itemize}


%-----------------------------------------------------------------------------%
\section{Posisi Penelitian}
%-----------------------------------------------------------------------------%
Sebutkan posisi penelitian Anda.

\begin{figure}
	\centering
	\includegraphics[width=\textwidth]{pics/makara.png}
	\caption{Penjelasan singkat terkait gambar.}
	\label{fig:research_position}
\end{figure}

Jelaskan \pic~\ref{fig:research_position} di sini.


%-----------------------------------------------------------------------------%
\section{Langkah Penelitian}
%-----------------------------------------------------------------------------%
Berikut ini adalah langkah penelitian yang telah dilakukan:
\begin{enumerate}
	\item Tinjauan literatur \\
	Pada tahap ini, dipelajari teori-teori yang terkait dengan penelitian ini untuk mendapatkan konsep dasar yang dibutuhkan dalam mencapai tujuan penelitian.
	\item Analisis implementasi dan kesimpulan \\
	Pada tahap ini, digunakan studi kasus untuk analisis terkait kegunaan \f{template}. Setelah melakukan analisis tersebut, ditarik kesimpulan keseluruhan dari penelitian ini.
\end{enumerate}


%-----------------------------------------------------------------------------%
\section{Sistematika Penulisan}
%-----------------------------------------------------------------------------%
Sistematika penulisan laporan adalah sebagai berikut:
\begin{itemize}
	\item Bab 1 \babSatu \\
	    Bab ini mencakup latar belakang, cakupan penelitian, dan pendefinisian masalah.
	\item Bab 2 \babDua \\
	    Bab ini mencakup pemaparan terminologi dan teori yang terkait dengan penelitian berdasarkan hasil tinjauan pustaka yang telah digunakan, sekaligus memperlihatkan kaitan teori dengan penelitian.
	\item Bab 3 \babTiga \\
	    Apa itu Bab 3?
	\item Bab 4 \babEmpat \\
		Apa itu Bab 4?
	\item Bab 5 \babLima \\
	    Apa itu Bab 5?
	\item Bab 6 \kesimpulan \\
	    Bab ini mencakup kesimpulan akhir penelitian dan saran untuk pengembangan berikutnya.
\end{itemize}

\clearchapter
%-----------------------------------------------------------------------------%
\chapter{\babDua}
%-----------------------------------------------------------------------------%
Untuk memulai penelitian, dibutuhkan kerangka berpikir yang sesuai untuk permasalahan yang ingin dipecahkan. Untuk membentuk kerangka berpikir yang sesuai, perlu dikaitkan dengan hasil studi literatur yang telah dilakukan. Oleh karena itu, pada bab ini, akan dijelaskan hasil studi literatur yang telah dilakukan yang telah dikaitan dengan kerangka kerja untuk penelitian ini.


%-----------------------------------------------------------------------------%
\section{\f{Sample Theory}}
%-----------------------------------------------------------------------------%
Ini teori saya \cite{book:test}.

\begin{figure}
	\centering
	\includegraphics[width=0.6\textwidth]{pics/makara.png}
	\captionsource{Makara UI}{\cite{book:sample}}
	\label{fig:sample}
\end{figure}

Jelaskan \pic~\ref{fig:sample}.


%-----------------------------------------------------------------------------%
\section{Keterkaitan Teori Dengan Penelitian}
%-----------------------------------------------------------------------------%
\begin{figure}
	\centering
	\includegraphics[width=0.8\textwidth]{pics/makara.png}
	\caption{Keterkaitan konsep hasil studi literatur terhadap penelitian}
	\label{fig:research_concept_map}
\end{figure}

Jelaskan \pic~\ref{fig:research_concept_map} di sini.

\clearchapter
%-----------------------------------------------------------------------------%
\chapter{\babTiga}
%-----------------------------------------------------------------------------%
Apa itu Bab 3?


%-----------------------------------------------------------------------------%
\section{Contoh}
%-----------------------------------------------------------------------------%
\begin{enumerate}
	\item Tahap 1. \\ \lstinputlisting[language=Java, caption=Contoh kode, label=code:sample]{codes/3-sample.java}
	Jelaskan \lst~\ref{code:sample} di sini.

	\item Tahap 2.
\end{enumerate}

Tulis kesimpulan di sini.

\clearchapter
%-----------------------------------------------------------------------------%
\chapter{\babEmpat}
%-----------------------------------------------------------------------------%
Bab ini menjelaskan tentang struktur dari \f{template} tugas akhir ini.
Dengan memahami struktur \f{template}, pekerjaan Anda akan menjadi lebih terarah karena Anda tahu di mana Anda harus melakukan sesuatu.

\todo{Sejatinya bab ini digunakan untuk membahas inti dari penelitian Anda. Sesuaikan saja dengan kebutuhkan Anda: misalkan bab empat Anda adalah penjelasan terkait implementasi sistem.}


%-----------------------------------------------------------------------------%
\section{thesis.tex}
%-----------------------------------------------------------------------------%
Berkas ini berisi seluruh berkas Latex yang dibaca, jadi bisa dikatakan sebagai berkas utama.
Dari berkas ini kita dapat mengatur bab apa saja yang ingin kita tampilkan dalam dokumen.


%-----------------------------------------------------------------------------%
\section{laporan\_setting.tex}
%-----------------------------------------------------------------------------%
Berkas ini berguna untuk mempermudah pembuatan beberapa template standar. 
Anda diminta untuk menuliskan judul laporan, nama, npm, dan hal-hal lain yang dibutuhkan untuk pembuatan template. 


%-----------------------------------------------------------------------------%
\section{istilah.tex}
%-----------------------------------------------------------------------------%
Berkas istilah digunakan untuk mencatat istilah-istilah yang digunakan. 
Fungsinya hanya untuk memudahkan penulisan.
Pada beberapa kasus, ada kata-kata yang harus selalu muncul dengan tercetak miring atau tercetak tebal. 
Dengan menjadikan kata-kata tersebut sebagai sebuah perintah \latex~tentu akan mempercepat dan mempermudah pengerjaan laporan. 


%-----------------------------------------------------------------------------%
\section{hype.indonesia.tex}
%-----------------------------------------------------------------------------%
Berkas ini berisi cara pemenggalan beberapa kata dalam bahasa Indonesia. 
\latex~memiliki algoritma untuk memenggal kata-kata sendiri, namun untuk beberapa kasus algoritma ini memenggal dengan cara yang salah. 
Untuk memperbaiki pemenggalan yang salah inilah cara pemenggalan yang benar ditulis dalam berkas \f{hype.indonesia.tex}.


%-----------------------------------------------------------------------------%
\section{pustaka.tex}
%-----------------------------------------------------------------------------%
Berkas pustaka.tex berisi seluruh daftar referensi yang digunakan dalam 
laporan. 
Anda bisa membuat model daftar referensi lain dengan menggunakan bibtex.
Untuk mempelajari bibtex lebih lanjut, silahkan buka \url{http://www.bibtex.org/Format}. 
Untuk merujuk pada salah satu referensi yang ada, gunakan perintah \bslash cite, e.g. \bslash cite\{book:sample\} yang akan akan memunculkan \cite{book:sample}.


%-----------------------------------------------------------------------------%
\section{bab[1 - 6].tex}
%-----------------------------------------------------------------------------%
Berkas ini berisi isi laporan yang Anda tulis. 
Setiap nama berkas e.g. bab1.tex merepresentasikan bab dimana tulisan tersebut akan muncul. 
Sebagai contoh, kode dimana tulisan ini dibaut berada dalam berkas dengan nama \code{bab4.tex}. 
Ada enam buah berkas yang telah disiapkan untuk mengakomodir enam bab dari laporan Anda, diluar bab kesimpulan dan saran. 
Jika Anda tidak membutuhkan sebanyak itu, silahkan hapus kode dalam berkas \code{thesis.tex} yang memasukan berkas \latex~yang tidak dibutuhkan; contohnya perintah \code{\bslash{}include\{bab6.tex\}} merupakan kode untuk memasukan berkas \code{bab6.tex} kedalam laporan.


\clearchapter
%-----------------------------------------------------------------------------%
\chapter{\babLima}
%-----------------------------------------------------------------------------%
Apa itu Bab 5?

\todo{Tuliskan paragraf pengantar bab lima di sini. Bab lima pada tugas akhir S1 umumnya merupakan pembahasan analisis dari penelitian. Namun, sekali lagi, sesuaikan dengan kebutuhan Anda. Tesis atau disertasi tentunya berbeda dengan skripsi.}


%-----------------------------------------------------------------------------%
\section{Metode Analisis}
%-----------------------------------------------------------------------------%
Metode analisis yang dilakukan dalam penelitian ini adalah sebagai berikut:
\begin{itemize}
	\item Tahap 1.
	\item Tahap 2.
\end{itemize}


%-----------------------------------------------------------------------------%
\section{Hasil Analisis}
%-----------------------------------------------------------------------------%
Berdasarkan analisis yang dilakukan pada \f{hardware} dengan spesifikasi ...., berikut ini adalah performa pembuatan produk yang dilakukan secara bersamaan:
\begin{table}
	\centering
	\begin{tabular}{|l|c|c|c|c|}
		\hline
		Jumlah   & \f{CPU Time} (s) & CPU \% / \f{core} & Waktu (s) & Memori (MB) \\ \hline
		1 produk & 101,21           & 35                & 37,86     & 917,35      \\ \hline
		5 produk & 120,728          & 17,18             & 87,46     & 915,28      \\ \hline
	\end{tabular}
	\caption{Contoh tabel analisis performa pembuatan produk ketika dijalankan bersamaan.}
	\label{table:sample}
\end{table}

\todo{Tulis penjelasan terkait \tab~\ref{table:sample} di sini. Jika Anda hanya menunjukkan data, pembaca tidak akan tahu apakah data tersebut berharga.}

\clearchapter
%---------------------------------------------------------------
\chapter{\kesimpulan}
%---------------------------------------------------------------
Pada bab ini, Penulis akan memaparkan kesimpulan penelitian dan saran untuk penelitian berikutnya.


%---------------------------------------------------------------
\section{Kesimpulan}
%---------------------------------------------------------------
Berikut ini adalah kesimpulan terkait pekerjaan yang dilakukan dalam penelitian ini:
\begin{enumerate}
	\item \bo{Poin pertama} \\
	Penjelasan poin pertama.
	\item \bo{Poin kedua} \\
	Penjelasan poin kedua.
\end{enumerate}

Tulis kalimat penutup di sini.


%---------------------------------------------------------------
\section{Saran}
%---------------------------------------------------------------
Berdasarkan hasil penelitian ini, berikut ini adalah saran untuk pengembangan penelitian berikutnya:
\begin{enumerate}
	\item Saran 1.
	\item Saran 2.
\end{enumerate}

\clearchapter

%
% Daftar Pustaka
%
% Daftar Pustaka
%

%
% Tambahkan pustaka yang digunakan setelah perintah berikut.
%
\phantomsection %hack to add clickable section for pustaka
\bibliography{config/references}

\clearchapter

%
% Lampiran
%
\begin{appendix}
	\include{markLampiran}
	\clearchapter
	\setcounter{page}{2}
	%-----------------------------------------------------------------------------%
\addChapter{Lampiran 1: Judul Lampiran 1}
\chapter*{Lampiran 1: Judul Lampiran 1}
%-----------------------------------------------------------------------------%
%-----------------------------------------------------------------------------%
\section*{Subbab dari Lampiran 1}
%-----------------------------------------------------------------------------%
Isi subbab dari lampiran 1.

\end{appendix}

\end{document}
