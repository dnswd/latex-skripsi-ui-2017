%
% Template Laporan Skripsi/Thesis Universitas Indonesia
%
% @author  Ichlasul Affan, Azhar Kurnia
% @version 2.1.0
%
% Dokumen ini dibuat berdasarkan standar IEEE dalam membuat class untuk
% LaTeX dan konfigurasi LaTeX yang digunakan Fahrurrozi Rahman ketika
% membuat laporan skripsi, yang kemudian diadaptasi oleh Andreas Febrian dan
% Lia Sadita untuk template skripsi tahun 2010.
% Konfigurasi template sebelumnya telah disesuaikan dengan
% aturan penulisan thesis yang dikeluarkan UI pada tahun 2017.
%

%
% Tipe dokumen adalah report dengan satu kolom.
%
\documentclass[12pt, a4paper, onecolumn, twoside, final]{report}
\raggedbottom

% Load konfigurasi LaTeX untuk tipe laporan thesis
\usepackage{_internals/uithesis}
%


% Load konfigurasi khusus untuk laporan yang sedang dibuat
%-----------------------------------------------------------------------------%
% Informasi Mengenai Dokumen
%-----------------------------------------------------------------------------%
%
% Judul laporan.
\def\judul{Judul Karya Ilmiah Anda}
%
% Tulis kembali judul laporan, kali ini akan diubah menjadi huruf kapital
\Var{\Judul}{Judul Karya Ilmiah Anda}
%
% Tulis kembali judul laporan namun dengan bahasa Ingris
\def\judulInggris{Your Scientific Publication Title}

%
% Tipe laporan, dapat berisi: Laporan Kerja Praktik, Skripsi, Tugas Akhir, Thesis, atau Disertasi
\def\type{Skripsi}
%
% Jenjang studi, dapat berisi: Diploma, Sarjana, Magister, atau Doktor
\def\jenjang{Sarjana}
%
% Tulis kembali tipe laporan, kali ini akan diubah menjadi huruf kapital
\Var{\Type}{Skripsi}
%
% Tulis nama penulis
\def\penulis{Nama Lengkap Anda}
%
% Tulis kembali nama penulis, kali ini akan diubah menjadi huruf kapital
\Var{\Penulis}{Nama Lengkap Anda}
%
% Tulis NPM penulis
\def\npm{Nomor Anda}
%
% Tuliskan Fakultas dimana penulis berada
\Var{\Fakultas}{Fakultas Anda}
\def\fakultas{Fakultas Anda}
%
% Tuliskan Program Studi yang diambil penulis
\Var{\Program}{Jurusan Anda}
\def\program{Jurusan Anda}
% Program Studi dalam bahasa inggris
\def\studyProgram{Your study program}
%
% Tuliskan tahun publikasi laporan
\Var{\bulanTahun}{Bulan Tahun}
%
% Tuliskan gelar yang akan diperoleh dengan menyerahkan laporan ini
\def\gelar{Gelar Jurusan Anda}
%
% Tuliskan tanggal pengesahan laporan, waktu dimana laporan diserahkan ke
% penguji/sekretariat
\def\tanggalSiapSidang{Tanggal Bulan Tahun}
%
% Tuliskan tanggal keputusan sidang dikeluarkan dan penulis dinyatakan
% lulus/tidak lulus
\def\tanggalLulus{Tanggal Bulan Tahun}
%
% Tuliskan pembimbing
\def\pembimbingSatu{Pembimbing Pertama Anda}
% S1 s.d. S3: Kosongkan jika tidak ada pembimbing kedua
\def\pembimbingDua{}
% S2 & S3: Kosongkan jika tidak ada pembimbing ketiga
\def\pembimbingTiga{}
%
% Tuliskan penguji
\def\pengujiSatu{Penguji Pertama Anda}
\def\pengujiDua{Penguji Kedua Anda}
% Kosongkan jika tidak ada penguji ketiga (umumnya penguji ketiga hanya ada untuk S2)
\def\pengujiTiga{}
% Kosongkan jika tidak ada penguji keempat, kelima, atau keenam (umumnya penguji > 3 hanya ada untuk S3)
\def\pengujiEmpat{}
\def\pengujiLima{}
\def\pengujiEnam{}

%-----------------------------------------------------------------------------%
% Judul Setiap Bab
%-----------------------------------------------------------------------------%
%
% Berikut ada judul-judul setiap bab.
% Silahkan diubah sesuai dengan kebutuhan.
%
\Var{\kataPengantar}{Kata Pengantar}
\Var{\babSatu}{Pendahuluan}
\Var{\babDua}{Kerangka Berpikir}
\Var{\babTiga}{Penggunaan Lanjutan}
\Var{\babEmpat}{Struktur Template}
\Var{\babLima}{Contoh Analisis dan Pembahasan}
\Var{\kesimpulan}{Penutup}

% Daftar pemenggalan suku kata dan istilah dalam LaTeX
%
% Hyphenation untuk Indonesia
%
% @author  Andreas Febrian
% @version 2.1.2
% @edit by Ichlasul Affan, Muhammad Aulia Adil Murtito
%
% Tambahkan cara pemenggalan kata-kata yang salah dipenggal secara otomatis
% oleh LaTeX. Jika kata tersebut dapat dipenggal dengan benar, maka tidak
% perlu ditambahkan dalam berkas ini. Tanda pemenggalan kata menggunakan
% tanda '-'; contoh:
% menarik
%   --> pemenggalan: me-na-rik
%


% Silakan ganti ke bahasa Inggris (\selectlanguage{english}) jika Anda merasa terlalu banyak kata bahasa Inggris yang pemenggalannya tidak benar.
%\selectlanguage{english}


\hyphenation{
    % alphabhet A
    a-na-li-sa a-tur a-tur-an
    a-pli-ka-si
    % alphabhet B
    bab ba-ngun-an
    be-be-ra-pa
    ber-ge-rak
    ber-ke-lan-jut-an
    ber-o-per-ra-si
    ber-pe-nga-ruh
    % alphabhet C
    ca-ri Com-po-nent-UML
    % alphabhet D
    di-da-pat-kan di-sim-pan di-pim-pin di-tam-bah-kan di-tem-pat-kan de-ngan da-e-rah di-ba-ngun di-gu-na-kan da-pat di-nya-ta-kan
    di-se-mat-kan di-sim-bol-kan di-pi-lih di-li-hat de-fi-ni-si di-de-fi-ni-si-kan di-mo-del-kan di-mi-li-ki di-re-a-li-sa-si-kan di-su-sun
    % alphabhet E
    eks-pli-sit e-ner-gi en-gi-neer en-gi-neer-ing eks-klu-sif ele-men
    % alphabhet F
    fa-si-li-tas
    % alphabhet G
    ga-bung-an ge-rak ge-ne-ral ge-ne-ra-li-sa-si
    % alphabhet H
    ha-lang-an
    % alphabhet I
    in-fra-struk-tur i-ni-si-a-si
    % alphabhet J
    % alphabhet K
    ke-hi-lang-an
    ke-ter-hu-bung-an
    ku-ning
    kua-li-tas ka-me-ra ke-mung-kin-an ke-se-pa-ham-an
    % alphabhet L
    ling-kung-an
    % alphabhet M
    ma-na-je-men me-neng-ah meng-a-da-kan me-mo-ni-tor
    me-mer-lu-kan me-mo-del-kan men-de-fi-ni-si-kan meng-ak-ses me-ne-mu-kan
    meng-a-tas-i me-mo-di-fi-ka-si me-mung-kin-kan me-nge-na-i me-ngi-rim-kan meng-i-zin-kan
    meng-u-bah meng-a-dap-ta-si me-nya-ta-kan me-nyim-pan me-res-trik-si mi-cro-ser-vi-ce mi-cro-ser-vi-ces mo-di-fi-ka-si mo-dul mo-dule
    meng-a-tur meng-a-rah-kan mi-lik meng-gu-na-kan me-ne-ri-ma me-nga-la-mi
    % alphabhet N
    nya-ta non-eks-klu-sif
    % alphabhet O
    o-pe-ra-si or-ga-ni-sa-si
    % alphabhet P
	pe-nye-rap-an
	pe-ngon-trol
    pe-mo-del-an
    pe-ran  pe-ran-an-nya
    pem-ba-ngun-an pre-si-den pe-me-rin-tah pe-mi-li-han prio-ri-tas peng-am-bil-an
    peng-ga-bung-an pe-nga-was-an pe-ngem-bang-an
    pe-nga-ruh pe-nge-lo-la pa-ra-lel-is-me per-hi-tung-an per-ma-sa-lah-an
    pen-ca-ri-an pen-ce-ta-kan peng-struk-tur-an pen-ting pen-ting-nya pe-ngu-ku-ran
    pre-sen-ta-si
    % alphabhet Q
    % alphabhet R
    ran-cang-an re-fe-ren-si re-pre-sen-ta-si
    % alphabhet S
    sub-bab si-mu-la-si sa-ngat ska-la-bi-li-tas
    stan-dar-di-sa-si
    % alphabhet T
    te-ngah
    ter-da-pat
    trans-for-ma-si
    % alphabhet U
    % alphabhet V
    va-li-da-si va-ri-an va-ri-a-si va-ri-a-bi-li-tas ve-ri-fi-ka-si
    % alphabhet W
    % alphabhet X
    % alphabhet Y
    % alphabhet Z
    % special
}

% Daftar istilah yang mungkin perlu ditandai
\input{config/istilah}

% Awal bagian penulisan laporan
\begin{document}
%
% Sampul Laporan
\include{_internals/sampul}
\forceclearchapter

%
% Gunakan penomeran romawi
\pagenumbering{roman}

%
% load halaman judul dalam
\addChapter{HALAMAN JUDUL}
\include{_internals/judul_dalam}
\forceclearchapter

%
% setelah bagian ini, halaman dihitung sebagai halaman ke 2
\setcounter{page}{2}

%
% load halaman orisinalitas
\addChapter{LEMBAR PERNYATAAN ORISINALITAS}
\include{src/00-front_matter/02-orisinal}
\forceclearchapter
%
%
\addChapter{LEMBAR PENGESAHAN}
\include{src/00-front_matter/03-pengesahan_sidang}
\forceclearchapter
%
%
\addChapter{\kataPengantar}
\include{src/00-front_matter/05-pengantar}
\forceclearchapter
%
%
\addChapter{LEMBAR PERSETUJUAN PUBLIKASI ILMIAH}
\include{src/00-front_matter/04-persetujuan_publikasi}
%

%
% Untuk halaman pertama setiap chapter mulai dari abstrak, tetap berikan mark universitas.
%
\pagestyle{first-pages}
\fancypagestyle{plain}{
	\fancyhead[L]{}
	\fancyhead[C]{}
	\fancyhead[R]{}
	\fancyfoot[L]{}
	\fancyfoot[C]{\thepage}
	\fancyfoot[R]{\footnotesize \fontfamily{phv} \selectfont \bo{Universitas Indonesia}}
}

%
\addChapter{ABSTRAK}
%
% Halaman Abstrak
%
% @author  Andreas Febrian
% @version 2.00
% @edit by Ichlasul Affan
%

\chapter*{Abstrak}
\singlespacing

\vspace*{0.2cm}

% Untuk conditional statement pembimbing dua
\def\blank{}

\noindent \begin{tabular}{l l p{10cm}}
	Nama&: & \penulis \\
	Program Studi&: & \program \\
	Judul&: & \judul \\
	Pembimbing&: & \pembimbingSatu \\
	\ifx\blank\pembimbingDua
    \else
        \ &\ & \pembimbingDua \\
    \fi
\end{tabular} \\

\vspace*{0.5cm}

\noindent Isi abstrak. \\

\vspace*{0.2cm}

\noindent Kata kunci: \\ \f{Keyword} satu, kata kunci dua \\

\onehalfspacing
\newpage

%
%
%
% Halaman Abstract
%
% @author  Andreas Febrian
% @version 2.00
% @edit by Ichlasul Affan
%

\chapter*{ABSTRACT}
\singlespacing

\vspace*{0.2cm}

% Untuk conditional statement pembimbing dua
\def\blank{}

\noindent \begin{tabular}{l l p{11.0cm}}
	Name&: & \penulis \\
	Study Program&: & \studyProgram \\
	Title&: & \judulInggris \\
	Counsellor&: & \pembimbingSatu \\
	\ifx\blank\pembimbingDua
    \else
        \ &\ & \pembimbingDua \\
    \fi
\end{tabular} \\

\vspace*{0.5cm}

\noindent Abstract content. \\

\vspace*{0.2cm}

\noindent Key words: \\ Keyword one, keyword two \\

\onehalfspacing
\newpage


%
% Daftar isi, gambar, tabel, dan kode
%
\CAPinToC % All entries in ToC will be CAPITALIZED from here on
\phantomsection %hack to make them clickable
\singlespacing
\tableofcontents
\onehalfspacing
\clearpage
\phantomsection %hack to make them clickable
\singlespacing
\listoffigures
\onehalfspacing
\clearpage
\phantomsection %hack to make them clickable
\singlespacing
\listoftables
\onehalfspacing
\clearpage

%
% Daftar Kode Program
% Comment to disable.
%
\phantomsection %hack to make them clickable
\addcontentsline{toc}{chapter}{\lstlistlistingname}
\singlespacing
\listoflistings
\onehalfspacing
\clearpage

%
% Daftar Isi yang Didefinisikan Sendiri (Custom)
% Definisi jenis objek baru dapat dilakukan di uithesis.sty
% Uncomment to use.
%
%\phantomsection %hack to make them clickable
%\addcontentsline{toc}{chapter}{\listofthingname}
%\singlespacing
%\listofthing
%\onehalfspacing
%\clearpage

%
% Daftar Equation (Persamaan Matematis)
% Uncomment to use.
%
% \phantomsection %hack to make them clickable
% \addcontentsline{toc}{chapter}{\listofequname}
% \singlespacing
% \listofequ
% \onehalfspacing
% \clearpage

%
% Daftar Lampiran
% Comment to disable.
%
\phantomsection %hack to make them clickable
\addcontentsline{toc}{chapter}{\listofappendixname}
\singlespacing
\listofappendix
\onehalfspacing
\clearpage

% Jika penomoran romawi selesai di ganjil
%\naiveoddclearchapter
% Jika penomoran romawi selesai di genap
%\naiveevenclearchapter

\noCAPinToC % Revert to original \addcontentsline formatting

%
% Gunakan penomeran Arab (1, 2, 3, ...) setelah bagian ini.
%
\pagenumbering{arabic}
\pagestyle{standard}
% \setlength{\belowcaptionskip}{+2pt}


\setoddevenheader
\include{src/01-body/01-bab1}
\clearchapter
%-----------------------------------------------------------------------------%
\chapter{\babDua}
\label{bab:2}
%-----------------------------------------------------------------------------%
Untuk memulai penelitian, dibutuhkan kerangka berpikir yang sesuai untuk permasalahan yang ingin dipecahkan. Untuk membentuk kerangka berpikir yang sesuai, perlu dikaitkan dengan hasil studi literatur yang telah dilakukan. Oleh karena itu, pada bab ini, akan dijelaskan hasil studi literatur yang telah dilakukan yang telah dikaitan dengan kerangka kerja untuk penelitian ini.


%-----------------------------------------------------------------------------%
\section{Apa itu \latex?}
\label{sec:latex}
%-----------------------------------------------------------------------------%

%-----------------------------------------------------------------------------%
\subsection{\latex~Secara Singkat}
\label{sec:latexBrief}
%-----------------------------------------------------------------------------%
Berdasarkan \cite{latex:intro}: \\
\begin{tabular}{| p{13cm} |}
	\hline
	\\
	LaTeX is a family of programs designed to produce publication-quality typeset documents. It is particularly strong when working with mathematical symbols. \\
	The history of LaTeX begins with a program called TEX. In 1978, a computer scientist by the name of Donald Knuth grew frustrated with the mistakes that his publishers made in typesetting his work. He decided to create a typesetting program that everyone could easily use to typeset documents, particularly those that include formulae, and made it freely available. The result is TEX. \\
	Knuth's product is an immensely powerful program, but one that does focus very much on small details. A mathematician and computer scientist by the name of Leslie Lamport wrote a variant of TEX called  that focuses on document structure rather than such details. \\
	\\
	\hline
\end{tabular}

\vspace*{0.8cm}

Dokumen \latex~sangat mudah, seperti halnya membuat dokumen teks biasa.
Ada beberapa perintah yang diawali dengan tanda '\bslash'.
Seperti perintah \code{\bslash\bslash}~yang digunakan untuk memberi baris baru.
Perintah tersebut juga sama dengan perintah \code{\bslash{}newline}.
Pada bagian ini akan sedikit dijelaskan cara manipulasi teks dan perintah-perintah \latex~yang mungkin akan sering digunakan.
Jika ingin belajar hal-hal dasar mengenai \latex, silakan kunjungi:

\begin{itemize}
	\item \url{http://frodo.elon.edu/tutorial/tutorial/}, atau
	\item \url{http://www.maths.tcd.ie/~dwilkins/LaTeXPrimer/}
\end{itemize}


%-----------------------------------------------------------------------------%
\subsection{\latex~Kompiler dan IDE}
\label{sec:latexCompiler}
%-----------------------------------------------------------------------------%
Untuk menggunakan \latex~(pada konteks hanya sebagai pengguna), tidak perlu banyak tahu mengenai hal-hal didalamnya.
Dengan menggunakan \f{Integrated Development Environment} (IDE), penggunaan \latex~akan serupa dengan pembuatan dokumen secara visual, layaknya OpenOffice Writer atau Microsoft Word.
Orang-orang yang menggunakan \latex~relatif lebih teliti dan terstruktur mengenai cara penulisan yang dia gunakan, karena \latex~memaksa untuk seperti itu.

Untuk mencoba \latex, diperlukan kompiler dan IDE.
Bagi pengguna Microsoft Windows dan Mac OS, instalasi kompiler \latex~dapat menggunakan MikTeX (\url{https://miktex.org/download}).
Bagi pengguna Linux, instalasi kompiler \latex~dapat menggunakan Texlive ( \url{http://www.tug.org/texlive/}).
Distro-distro \f{mainstream} di Linux seperti Ubuntu biasanya telah menyediakan \f{package} \code{texlive} melalui \f{package manager}.
Apabila ingin melakukan instalasi Texlive melalui \f{package manager}, lakukan instalasi package \code{texlive-full} atau setidaknya \code{texlive-science} agar prasyarat \f{template} ini tersedia secara lengkap.

Beberapa text editor atau IDE yang dapat digunakan adalah sebagai berikut:
\begin{itemize}
	\item TeXstudio (\url{https://www.texstudio.org/}).
	\item TeXWorks (biasanya bawaan dari MikTeX).
	\item Texmaker (\url{http://www.xm1math.net/texmaker/}).
	\item Microsoft Visual Studio Code, dengan \f{plugin} LaTeX Workshop (\url{https://marketplace.visualstudio.com/items?itemName=James-Yu.latex-workshop}). Untuk menggunakan \f{plugin} tersebut, diperlukan instalasi MikTeX dan Perl. Alternatif lain untuk persyaratan tersebut adalah menggunakan \f{plugin} Remote - WSL jika memiliki distro Windows Subsystem for Linux (WSL) 2 yang sudah terpasang \code{texlive}.
\end{itemize}


%-----------------------------------------------------------------------------%
\section{Panduan Pengunaan Dasar \latex}
\label{sec:latexUsage}
%-----------------------------------------------------------------------------%

%-----------------------------------------------------------------------------%
\subsection{Bold, Italic, dan Underline}
\label{sec:latexBIU}
%-----------------------------------------------------------------------------%
Hal pertama yang mungkin ditanyakan adalah bagaimana membuat huruf tercetak tebal, miring, atau memiliki garis bawah.
Pada Texmaker, Anda bisa melakukan hal ini seperti halnya saat mengubah dokumen dengan OO Writer.
Namun jika tetap masih tertarik dengan cara lain, ini dia:

\begin{itemize}
	\item \bo{Bold} \\
	Gunakan perintah \code{\bslash{}textbf$\lbrace\rbrace$} atau
	\code{\bslash{}bo$\lbrace\rbrace$}.
	\item \f{Italic} \\
	Gunakan perintah \code{\bslash{}textit$\lbrace\rbrace$} atau
	\code{\bslash{}f$\lbrace\rbrace$}.
	\item \underline{Underline} \\
	Gunakan perintah \code{\bslash{}underline$\lbrace\rbrace$}.
	\item $\overline{Overline}$ \\
	Gunakan perintah \code{\bslash{}overline}.
	\item $^{superscript}$ \\
	Gunakan perintah \code{\bslash{}$\lbrace\rbrace$}.
	\item $_{subscript}$ \\
	Gunakan perintah \code{\bslash{}\_$\lbrace\rbrace$}.
\end{itemize}

Perintah \code{\bslash{}f} dan \code{\bslash{}bo} hanya dapat digunakan jika package \code{uithesis} digunakan.


%-----------------------------------------------------------------------------%
\subsection{Memasukan Gambar}
\label{sec:latexImage}
%-----------------------------------------------------------------------------%
Setiap gambar dapat diberikan caption dan diberikan label. Label dapat digunakan untuk menunjuk gambar tertentu.
Jika posisi gambar berubah, maka nomor gambar juga akan diubah secara
otomatis.
Begitu juga dengan seluruh referensi yang menunjuk pada gambar tersebut.
Contoh sederhana adalah \pic~\ref{fig:testGambar}.
Silahkan lihat code \latex~dengan nama bab2.tex untuk melihat kode lengkapnya.
Harap diingat bahwa caption untuk gambar selalu terletak dibawah gambar.

\begin{figure}
	\centering
	\includegraphics[width=0.50\textwidth]
	{assets/pics/creative_commons.png}
	\caption{\license.}
	\label{fig:testGambar}
\end{figure}


%-----------------------------------------------------------------------------%
\section{Membuat Tabel}
\label{sec:latexTable}
%-----------------------------------------------------------------------------%
Tabel pada Latex dapat dibuat dengan bantuan \textit{website} seperti \url{https://www.tablesgenerator.com/}. Dengan menggunakan \textit{website} ini, maka pembuatan tabel akan menjadi lebih mudah. \textit{User interface} dari \url{https://www.tablesgenerator.com/} dapat dilihat pada Gambar \ref{fig:tablesgenerator}.

\begin{figure}
	\centering
	\includegraphics[width=0.5\textwidth]{assets/pics/tablesgenerator-dot-com.png}
	\caption{\textit{User interface} dari \textit{website} https://www.tablesgenerator.com/}
	\label{fig:tablesgenerator}
\end{figure}

Di sisi lain, tabel juga dapat diberi label dan caption seperti pada gambar.
Caption pada tabel terletak pada bagian atas tabel.
Contoh tabel sederhana dapat dilihat pada \tab~\ref{tab:tab1}.

\begin{table}
	\centering
	\caption{Contoh Tabel}
	\label{tab:tab1}
	\begin{tabular}{| l | c r |}
		\hline
		& kol 1 & kol 2 \\
		\hline
		baris 1 & 1 & 2 \\
		baris 2 & 3 & 4 \\
		baris 3 & 5 & 6 \\
		baris 4 & 7 & 8 \\
		baris 5 & 9 & 10 \\
		\hline
		jumlah  & 25 & 30 \\
		\hline
	\end{tabular}
\end{table}

Adapun untuk membuat tabel panjang yang bisa melebihi dari satu halaman, gunakan perintah \code{\bslash{}begin\{longtable\}} sebagai pengganti \code{\bslash{}begin\{table\}}. Di dalam \code{longtable} tidak perlu lagi ada \code{\bslash{}begin\{tabular\}}. Kemudian, tambahkan tanda \code{\bslash{}\bslash{}} setelah baris \code{\bslash{}label\{....\}}, agar tidak menimbulkan error saat menampilkan \f{caption} di bagian atas tabel. Kemudian, untuk membatasi header yang ingin diulang pada halaman-halaman berikutnya, gunakan perintah \code{\bslash{}endhead}. Contohnya adalah sebagai berikut:

\begin{longtable}{| l | c r |}
\caption{Contoh Tabel Panjang}
\label{tab:tab2} \\
\hline
& kol 1 & kol 2 \\
\hline
\endfirsthead % batas akhir header yang akan muncul di halaman pertama
\hline
& kol 1 & kol 2 \\
\hline
\endhead % batas akhir header yang akan muncul di halaman berikutnya
baris 1  & 1 & 2 \\
baris 2  & 3 & 4 \\
baris 3  & 5 & 6 \\
baris 4  & 7 & 8 \\
baris 5  & 9 & 10 \\
baris 6  & 11 & 12 \\
baris 7  & 13 & 14 \\
baris 8  & 15 & 16 \\
baris 9  & 17 & 18 \\
baris 10 & 19 & 20 \\
baris 11 & 21 & 22 \\
baris 12 & 23 & 24 \\
baris 13 & 25 & 26 \\
baris 14 & 27 & 28 \\
baris 15 & 29 & 30 \\
\hline
\end{longtable}

Ada jenis tabel lain yang dapat dibuat dengan \latex~berikut beberapa diantaranya.
Contoh-contoh ini bersumber dari \url{http://en.wikibooks.org/wiki/LaTeX/Tables}

\begin{table}
	\centering
	\caption{An Example of Rows Spanning Multiple Columns}
	\label{row.spanning}
	\begin{tabular}{|l|l|*{6}{c|}}
		\hline % create horizontal line
		No & Name & \multicolumn{3}{|c|}{Week 1} & \multicolumn{3}{|c|}{Week 2} \\
		\cline{3-8} % create line from 3rd column till 8th column
		& & A & B & C & A & B & C\\
		\hline
		1 & Lala & 1 & 2 & 3 & 4 & 5 & 6\\
		2 & Lili & 1 & 2 & 3 & 4 & 5 & 6\\
		3 & Lulu & 1 & 2 & 3 & 4 & 5 & 6\\
		\hline
	\end{tabular}
\end{table}

\begin{table}
	\centering
	\caption{An Example of Columns Spanning Multiple Rows}
	\label{column.spanning}
	\begin{tabular}{|l|c|l|}
		\hline
		Percobaan & Iterasi & Waktu \\
		\hline
		Pertama & 1 & 0.1 sec \\ \hline
		\multirow{2}{*}{Kedua} & 1 & 0.1 sec \\
		& 3 & 0.15 sec \\
		\hline
		\multirow{3}{*}{Ketiga} & 1 & 0.09 sec \\
		& 2 & 0.16 sec \\
		& 3 & 0.21 sec \\
		\hline
	\end{tabular}
\end{table}

\begin{table}
	\centering
	\caption{An Example of Spanning in Both Directions Simultaneously}
	\label{mix.spanning}
	\begin{tabular}{cc|c|c|c|c|}
		\cline{3-6}
		& & \multicolumn{4}{|c|}{Title} \\ \cline{3-6}
		& & A & B & C & D \\ \hline
		\multicolumn{1}{|c|}{\multirow{2}{*}{Type}} &
		\multicolumn{1}{|c|}{X} & 1 & 2 & 3 & 4\\ \cline{2-6}
		\multicolumn{1}{|c|}{}                        &
		\multicolumn{1}{|c|}{Y} & 0.5 & 1.0 & 1.5 & 2.0\\ \cline{1-6}
		\multicolumn{1}{|c|}{\multirow{2}{*}{Resource}} &
		\multicolumn{1}{|c|}{I} & 10 & 20 & 30 & 40\\ \cline{2-6}
		\multicolumn{1}{|c|}{}                        &
		\multicolumn{1}{|c|}{J} & 5 & 10 & 15 & 20\\ \cline{1-6}
	\end{tabular}
\end{table}


%-----------------------------------------------------------------------------%
\section{Keterkaitan Teori Dengan Penelitian}
\label{sec:keterkaitan}
%-----------------------------------------------------------------------------%
\todo{Ada baiknya setelah menjelaskan teori-teori, Anda menjelaskan apa kaitan teori tersebut dengan penelitian Anda. Hal ini tentunya membantu pembaca dalam memahami bahwa teori yang Anda paparkan memang penting untuk memahami penelitian Anda nantinya.}

\begin{figure}
	\centering
	\includegraphics[width=\textwidth]{assets/pics/research_concept_map.png}
	\caption{Keterkaitan konsep hasil studi literatur terhadap penelitian}
	\label{fig:research_concept_map}
\end{figure}

\todo{Jelaskan \pic~\ref{fig:research_concept_map} di sini. Setiap gambar pada tugas akhir butuh penjelasan. Gambar hadir untuk mempermudah membaca memahami konteks, tetapi tidak bisa berdiri sendiri tanpa penjelasan. Terkait gambar, Anda juga bisa mengatur skalanya. Gambar kali ini lebarnya 0,8x dari lebar teks halaman.}

\clearchapter
\include{src/01-body/03-bab3}
\clearchapter
\include{src/01-body/04-bab4}
\clearchapter
\include{src/01-body/05-bab5}
\clearchapter
\include{src/01-body/99-kesimpulan}
\clearchapter

%
% Daftar Pustaka
\CAPinToC % All entries in ToC will be CAPITALIZED from here on
%
% Daftar Pustaka
%

%
% Tambahkan pustaka yang digunakan setelah perintah berikut.
%
\phantomsection %hack to add clickable section for pustaka
\bibliography{config/references}

\clearchapter
\noCAPinToC % Revert to original \addcontentsline formatting

%
% Lampiran
%
\begin{appendix}
	\newcounter{pagetemp}
	\setcounter{pagetemp}{\thepage}
	\include{_internals/markLampiran}
	\clearchapter
	\setcounter{page}{\thepagetemp}
	\stepcounter{page}
	\include{src/99-back_matter/01-lampiran}
\end{appendix}

\end{document}
